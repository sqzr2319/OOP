\documentclass[a4paper, 12pt]{article}
\usepackage{ctex}

% 字体设置
\usepackage{fontspec}
\usepackage{xeCJK}

% 在导言区添加以下内容(文档开头处)
\usepackage{listings}
\usepackage{xcolor}

\usepackage{graphicx}

\usepackage{url}

\usepackage{fancyhdr}

% 配置listings包
\lstset{
    breaklines=true,           % 自动换行
    breakatwhitespace=false,   % 允许在任何位置换行,不仅是空白处
    numbers=none,              % 不显示行号
    basicstyle=\ttfamily\small, % 基本样式使用等宽字体
    commentstyle=\color[RGB]{0,128,0},  % 注释样式 - 绿色
    keywordstyle=\color[RGB]{0,0,255},  % 关键词样式 - 蓝色
    stringstyle=\color[RGB]{163,21,21}, % 字符串样式 - 红棕色
    numberstyle=\color[RGB]{9,134,88},  % 数字样式 - 绿色
    identifierstyle=\color{black},      % 标识符样式
    showstringspaces=false,     % 不显示字符串中的空格
    tabsize=4,                  % tab为4个空格
    emphstyle=\color[RGB]{38,127,153}, % 强调样式 - 青色(类型)
}

% 为C++代码定义专门的样式
\lstdefinestyle{cpp}{
    language=C++,
    morekeywords={constexpr, nullptr, auto, noexcept, static_assert, 
                  emplace, sorted_unique_t, sorted_t}
}

\setmonofont{Consolas}

\setCJKmainfont{黑体}
\setCJKsansfont{黑体}
\setCJKmonofont{黑体}


\title{std::flat\_map与std::flat\_multimap的用法}
\author{褚一枫\ 2024013328\ 17612197488\\zhuyf24@mails.tsinghua.edu.cn}
\date{\today}

\begin{document}

\pagenumbering{roman}

\sloppy
\maketitle

\begin{abstract}
    C++24\ 标准引入了两种新的容器:std::flat\_map与std::flat\_multimap,它们与传统容器std::map与std::multimap具有几乎完全相同的接口,但其底层实现的不同造成了性能上的差异。本文总结了std::flat\_map与std::flat\_multimap的相关用法,通过编写使用案例分析了使用时的注意事项,并通过设计实验与理论分析,对其性能进行了评估,从而分析了其优缺点与使用场景。
\end{abstract}
\textbf{关键词:C++24, std::flat\_map, std::flat\_multimap, 性能分析}

~

\noindent\textbf{所有实验都在相同的硬件和编译器环境下运行,使用以下通用配置:}
\begin{itemize}
    \setlength{\itemsep}{0pt}
    \setlength{\parsep}{0pt}
    \setlength{\parskip}{0pt}
    \item 处理器:AMD Ryzen AI 9 HX 370 
    \item 内存:32GB LPDDR5X
    \item 编译器:g++-15 (SUSE Linux) 15.0.1 20250317 (experimental)
    \item 操作系统:openSUSE Tumbleweed 20250316
    \item 编程语言:C++24
    \item 编译环境:Windows Subsystem for Linux (WSL)
    \item 编译参数:-O3 -std=c++23
\end{itemize}

\noindent{参考文献电子版见ref文件夹,实验源代码见src文件夹。}

\clearpage

\tableofcontents

\clearpage
\pagenumbering{arabic}  % 切换到阿拉伯数字页码并重置为1
\setcounter{page}{1}

\section{简介}

C++作为一种广泛应用于系统开发、游戏引擎、高性能应用等领域的编程语言,其标准库的不断演进为开发者提供了越来越丰富的工具。C++24标准作为最新的C++标准,引入了多项重要更新,其中就包括两个新的关联容器:std::flat\_map与std::flat\_multimap。

在传统C++中,关联容器std::map和std::multimap通常基于红黑树或其他平衡二叉搜索树实现,这种结构能够保证对数级别的查找、插入和删除时间复杂度。然而,红黑树结构内存分配次数多且数据在内存中分散存储,这可能导致缓存不友好,影响实际应用中的性能表现。

std::flat\_map和std::flat\_multimap采用了不同的设计理念,它们使用两个平行数组(通常是std::vector)分别存储键和值,一个用于所有键,另一个用于所有对应的值。这种扁平化的存储结构带来了几个显著特点:
\begin{itemize}
    \setlength{\itemsep}{0pt}
    \setlength{\parsep}{0pt}
    \setlength{\parskip}{0pt}
    \item 内存连续性更好,提高了缓存友好性
    \item 在某些场景下可能提供更好的性能
    \item 提供了与传统map容器几乎相同的接口,便于开发者迁移现有代码
\end{itemize}

然而,这种实现方式在插入和删除操作时可能需要移动大量元素,从而导致性能下降。因此,了解这些新容器的行为特性、性能特点以及适用场景对于开发者来说至关重要。

本文将深入探讨std::flat\_map与std::flat\_multimap的基本用法、实现原理、性能特性以及适用场景。首先介绍其API及使用方式,然后通过具体案例展示其实际应用,进而通过实验和理论分析对比评估其性能特点,最后总结其优缺点并给出合适的使用场景建议。通过本文,读者将能够全面了解这两种新容器,并在实际开发中做出明智的容器选择。

\section{相关用法}

本节将详细介绍std::flat\_map与std::flat\_multimap的基本用法,包括头文件包含、构造方法、基本操作以及特有的成员函数等。

\subsection{基本概念与头文件}

std::flat\_map和std::flat\_multimap定义在C++23标准库中,需要包含对应的头文件:

\begin{lstlisting}[style=cpp]
#include <flat_map>     // 对于std::flat_map
#include <flat_map>     // 对于std::flat_multimap(同一头文件)
\end{lstlisting}

这两个容器的基本概念如下:
\begin{itemize}
    \setlength{\itemsep}{0pt}
    \setlength{\parsep}{0pt}
    \setlength{\parskip}{0pt}
    \item \textbf{std::flat\_map}:不允许重复键的关联容器,使用两个连续存储的序列容器分别存储键和值
    \item \textbf{std::flat\_multimap}:允许重复键的关联容器,底层实现与flat\_map类似
\end{itemize}

\subsection{模板参数}

std::flat\_map与std::flat\_multimap的模板参数定义如下:

\begin{lstlisting}[style=cpp]
template<
    class Key,                             // 键类型
    class T,                               // 值类型
    class Compare = std::less<Key>,        // 比较函数对象类型
    class KeyContainer = std::vector<Key>, // 存储键的容器类型
    class MappedContainer = std::vector<T> // 存储值的容器类型
> class flat_map;

template<
    class Key,
    class T,
    class Compare = std::less<Key>,
    class KeyContainer = std::vector<Key>,
    class MappedContainer = std::vector<T>
> class flat_multimap;
\end{lstlisting}

值得注意的是,与传统的map不同,这里允许用户指定用于存储键和值的底层容器类型,默认为std::vector。

\subsection{构造方法}

flat\_map和flat\_multimap提供了多种构造方法:

\begin{lstlisting}[style=cpp]
// 默认构造函数
flat_map();

// 使用比较器构造
explicit flat_map(const Compare& comp);

// 使用初始化列表构造
flat_map(std::initializer_list<std::pair<Key, T>> init);

// 从键值对范围构造
template<class InputIt>
flat_map(InputIt first, InputIt last);

// 从排序范围构造(sorted_unique_t标记表示输入已排序且唯一)
template<class InputIt>
flat_map(sorted_unique_t, InputIt first, InputIt last);

// 从键容器和值容器构造
flat_map(KeyContainer&& keys, MappedContainer&& maps);
\end{lstlisting}

flat\_multimap的构造方法与flat\_map类似,但在接受已排序范围时使用的是sorted\_t标记而非sorted\_unique\_t。

\subsection{基本操作}

\subsubsection{元素访问}

\begin{lstlisting}[style=cpp]
// 访问指定键的元素,若不存在则插入默认值
T& operator[](const Key& key);
T& operator[](Key&& key);

// 访问指定键的元素,若不存在则抛出异常
T& at(const Key& key);
const T& at(const Key& key) const;
\end{lstlisting}

需要注意的是,std::flat\_multimap不提供operator[]操作,因为它允许存在重复键。

\subsubsection{迭代器}

\begin{lstlisting}[style=cpp]
// 返回指向容器第一个元素的迭代器
iterator begin() noexcept;
const_iterator begin() const noexcept;

// 返回指向容器尾部的迭代器
iterator end() noexcept;
const_iterator end() const noexcept;

// 返回反向迭代器
reverse_iterator rbegin() noexcept;
const_reverse_iterator rbegin() const noexcept;
reverse_iterator rend() noexcept;
const_reverse_iterator rend() const noexcept;
\end{lstlisting}

\subsubsection{容量操作}

\begin{lstlisting}[style=cpp]
// 检查容器是否为空
[[nodiscard]] bool empty() const noexcept;

// 返回容器中的元素数量
size_type size() const noexcept;

// 返回容器可容纳的最大元素数量
size_type max_size() const noexcept;
\end{lstlisting}

\subsubsection{修改操作}

\begin{lstlisting}[style=cpp]
// 清空容器
void clear() noexcept;

// 插入元素
std::pair<iterator, bool> insert(const value_type& value);
std::pair<iterator, bool> insert(value_type&& value);

// 原位构造元素
template<class... Args>
std::pair<iterator, bool> emplace(Args&&... args);

// 删除元素
iterator erase(const_iterator pos);
iterator erase(const_iterator first, const_iterator last);
size_type erase(const Key& key);

// 交换内容
void swap(flat_map& other) noexcept;
\end{lstlisting}

flat\_multimap的insert方法返回iterator而非std::pair<iterator, bool>,因为它允许插入重复键。

\subsubsection{查找操作}

\begin{lstlisting}[style=cpp]
// 查找具有指定键的元素
iterator find(const Key& key);
const_iterator find(const Key& key) const;

// 检查容器是否包含具有特定键的元素
bool contains(const Key& key) const;

// 返回指定键的元素数量
size_type count(const Key& key) const;

// 返回第一个不小于给定键的元素的迭代器
iterator lower_bound(const Key& key);
const_iterator lower_bound(const Key& key) const;

// 返回第一个大于给定键的元素的迭代器
iterator upper_bound(const Key& key);
const_iterator upper_bound(const Key& key) const;

// 返回键值等于给定键的元素范围
std::pair<iterator, iterator> equal_range(const Key& key);
std::pair<const_iterator, const_iterator> equal_range(const Key& key) const;
\end{lstlisting}

\subsection{特有操作}

std::flat\_map和std::flat\_multimap相比传统map,还具有一些特有的操作:

\begin{lstlisting}[style=cpp]
// 获取键容器的引用
const key_container_type& keys() const noexcept;

// 获取值容器的引用
const mapped_container_type& values() const noexcept;

// 提取底层容器
std::pair<key_container_type, mapped_container_type> extract() &&;

// 替换底层容器
void replace(key_container_type&& key_cont, mapped_container_type&& mapped_cont);
\end{lstlisting}

这些特有操作反映了flat\_map和flat\_multimap的实现特点,允许用户直接操作底层容器,或执行特定的排序和去重操作。

\subsection{自定义比较器}

与std::map类似,std::flat\_map和std::flat\_multimap也支持自定义比较器:

\begin{lstlisting}[style=cpp]
struct CustomCompare {
    bool operator()(const Key& a, const Key& b) const {
        // 自定义比较逻辑
        return a < b;
    }
};

std::flat_map<Key, Value, CustomCompare> myMap;
\end{lstlisting}

通过指定自定义比较器,用户可以控制元素在容器中的排序方式。

\section{使用案例}

本节将通过具体的代码示例展示std::flat\_map和std::flat\_multimap的实际应用,以及使用时的注意事项。

\subsection{基本使用示例}

以下是一个简单的std::flat\_map使用示例,展示了其基本操作:

\begin{lstlisting}[style=cpp]
#include <flat_map>
#include <iostream>
#include <string>

int main() {
    // 创建并初始化flat_map
    std::flat_map<int, std::string> studentIds = {
        {1001, "张三"},
        {1002, "李四"},
        {1003, "王五"}
    };

    // 使用operator[]添加新元素
    studentIds[1004] = "赵六";

    // 使用emplace添加新元素
    studentIds.emplace(1005, "钱七");

    // 访问元素
    std::cout << "学号1003对应学生: " << studentIds[1003] << std::endl;

    // 检查键是否存在
    if (studentIds.contains(1006)) {
        std::cout << "学号1006存在" << std::endl;
    } else {
        std::cout << "学号1006不存在" << std::endl;
    }

    // 遍历所有元素
    std::cout << "所有学生信息:" << std::endl;
    for (const auto& [id, name] : studentIds) {
        std::cout << "学号: " << id << ", 姓名: " << name << std::endl;
    }

    return 0;
}
\end{lstlisting}

运行结果如下:

\begin{lstlisting}[style=cpp]
学号1003对应学生: 王五
学号1006不存在
所有学生信息:
学号: 1001, 姓名: 张三
学号: 1002, 姓名: 李四
学号: 1003, 姓名: 王五
学号: 1004, 姓名: 赵六
学号: 1005, 姓名: 钱七
\end{lstlisting}

\subsection{使用自定义键类型}

以下示例展示了如何使用自定义类型作为键,以及如何提供自定义比较器:

\begin{lstlisting}[style=cpp]
#include <flat_map>
#include <iostream>
#include <string>

// 自定义坐标类型
struct Point {
    int x, y;

    // 便于输出的友元函数
    friend std::ostream& operator<<(std::ostream& os, const Point& p) {
        return os << "(" << p.x << ", " << p.y << ")";
    }
};

// 自定义比较器,按点的曼哈顿距离排序
struct PointCompare {
    bool operator()(const Point& a, const Point& b) const {
        // 计算到原点的曼哈顿距离
        int distA = std::abs(a.x) + std::abs(a.y);
        int distB = std::abs(b.x) + std::abs(b.y);
        
        if (distA == distB) {
            // 距离相等时先比较x再比较y
            return (a.x == b.x) ? (a.y < b.y) : (a.x < b.x);
        }
        return distA < distB;
    }
};

int main() {
    // 使用自定义键类型和比较器
    std::flat_map<Point, std::string, PointCompare> locations;
    
    // 添加一些位置
    locations.insert({{3, 4}, "传感器A"});
    locations.insert({{1, 1}, "传感器B"});
    locations.insert({{5, 0}, "传感器C"});
    locations.insert({{-2, 3}, "传感器D"});
    
    // 遍历元素(将按自定义比较器排序)
    std::cout << "按到原点距离排序的位置:" << std::endl;
    for (const auto& [pos, name] : locations) {
        std::cout << name << " 位置: " << pos << std::endl;
    }
    
    return 0;
}
\end{lstlisting}

运行结果如下:

\begin{lstlisting}[style=cpp]
按到原点距离排序的位置:
传感器B 位置: (1, 1)
传感器D 位置: (-2, 3)
传感器C 位置: (5, 0)
传感器A 位置: (3, 4)
\end{lstlisting}

\subsection{利用特有操作}

以下示例展示了flat\_map特有的操作,如直接访问底层容器功能:

\begin{lstlisting}[style=cpp]
#include <flat_map>
#include <iostream>
#include <vector>

int main() {
    // 创建flat_map
    std::flat_map<int, double> dataPoints = {
        {5, 10.5}, {3, 7.2}, {8, 12.3}, {1, 5.1}
    };
    
    // 直接访问键容器和值容器
    std::cout << "键容器内容:" << std::endl;
    for (const auto& key : dataPoints.keys()) {
        std::cout << key << " ";
    }
    std::cout << std::endl;
    
    std::cout << "值容器内容:" << std::endl;
    for (const auto& value : dataPoints.values()) {
        std::cout << value << " ";
    }
    std::cout << std::endl;
    
    // 提取底层容器
    auto [keys, values] = std::move(dataPoints).extract();
    
    // 手动创建新的容器
    std::vector<int> newKeys = {10, 20, 30};
    std::vector<double> newValues = {100.1, 200.2, 300.3};
    
    // 使用新容器创建flat_map
    std::flat_map<int, double> newDataPoints(std::move(newKeys), std::move(newValues));
    
    // 在插入无序元素后使用手动排序
    std::flat_map<int, std::string> events;
    events.emplace(50, "中点事件");
    events.emplace(10, "起始事件");
    events.emplace(90, "结束事件");
    events.emplace(30, "早期事件");
    events.emplace(70, "后期事件");
    
    // 遍历排序后的元素
    std::cout << "排序后的事件:" << std::endl;
    for (const auto& [time, desc] : events) {
        std::cout << "时间: " << time << ", 描述: " << desc << std::endl;
    }
    
    return 0;
}
\end{lstlisting}

运行结果如下:

\begin{lstlisting}[style=cpp]
键容器内容:
1 3 5 8
值容器内容:
5.1 7.2 10.5 12.3
排序后的事件:
时间: 10, 描述: 起始事件
时间: 30, 描述: 早期事件
时间: 50, 描述: 中点事件
时间: 70, 描述: 后期事件
时间: 90, 描述: 结束事件
\end{lstlisting}

\subsection{大批量数据的预分配}

当预先知道容器大小时,可以利用底层容器的预分配能力提高性能:

\begin{lstlisting}[style=cpp]
#include <flat_map>
#include <chrono>
#include <iostream>
#include <vector>
#include <random>

int main() {
    constexpr int DATA_SIZE = 10000;
    
    // 创建随机数生成器
    std::random_device rd;
    std::mt19937 gen(rd());
    std::uniform_int_distribution<> keyDist(1, DATA_SIZE * 10);
    
    // 准备数据
    std::vector<int> keys;
    std::vector<double> values;
    
    keys.reserve(DATA_SIZE);    // 提前分配空间
    values.reserve(DATA_SIZE);  // 提前分配空间
    
    // 生成随机数据
    for (int i = 0; i < DATA_SIZE; ++i) {
        keys.push_back(keyDist(gen));
        values.push_back(static_cast<double>(i) / 10.0);
    }
    
    // 使用预分配的容器构造flat_map
    auto start = std::chrono::high_resolution_clock::now();
    std::flat_map<int, double> dataMap(std::move(keys), std::move(values));
    auto end = std::chrono::high_resolution_clock::now();
    
    std::chrono::duration<double, std::milli> duration = end - start;
    std::cout << "构造并排序包含 " << dataMap.size() 
              << " 个元素的flat_map耗时: " << duration.count() 
              << " 毫秒" << std::endl;
    
    return 0;
}
\end{lstlisting}

运行结果如下:

\begin{lstlisting}[style=cpp]
构造并排序包含 9537 个元素的flat_map耗时: 0.408881 毫秒
\end{lstlisting}

\subsection{使用flat\_multimap管理重复键}

以下示例展示了如何使用flat\_multimap处理具有重复键的数据:

\begin{lstlisting}[style=cpp]
#include <flat_map>
#include <iostream>
#include <string>

int main() {
    // 创建flat_multimap存储学生课程信息
    std::flat_multimap<std::string, std::string> studentCourses;
    
    // 插入数据(每个学生可以选多门课程)
    studentCourses.insert({"李明", "数学"});
    studentCourses.insert({"张华", "物理"});
    studentCourses.insert({"李明", "英语"});  // 李明的第二门课
    studentCourses.insert({"王芳", "化学"});
    studentCourses.insert({"张华", "生物"});  // 张华的第二门课
    studentCourses.insert({"李明", "历史"});  // 李明的第三门课
    
    // 按学生分组打印选课情况
    std::string currentStudent;
    std::cout << "学生选课情况:" << std::endl;
    
    for (const auto& [student, course] : studentCourses) {
        if (student != currentStudent) {
            currentStudent = student;
            std::cout << "\n" << student << " 选修的课程:";
        }
        std::cout << " " << course;
    }
    std::cout << std::endl;
    
    // 查找特定学生的所有课程
    std::string targetStudent = "李明";
    auto [begin, end] = studentCourses.equal_range(targetStudent);
    
    std::cout << "\n使用equal_range查找 " << targetStudent << " 的所有课程:";
    for (auto it = begin; it != end; ++it) {
        std::cout << " " << it->second;
    }
    std::cout << std::endl;
    
    // 计算特定学生的课程数量
    std::cout << targetStudent << " 总共选修了 " 
              << studentCourses.count(targetStudent) << " 门课程" << std::endl;
    
    return 0;
}
\end{lstlisting}

运行结果如下:

\begin{lstlisting}[style=cpp]
学生选课情况:

张华 选修的课程: 物理 生物
李明 选修的课程: 数学 英语 历史
王芳 选修的课程: 化学

使用equal_range查找 李明 的所有课程: 数学 英语 历史
李明 总共选修了 3 门课程
\end{lstlisting}

\subsection{使用注意事项}

在使用std::flat\_map和std::flat\_multimap时,需要注意以下几点:

\begin{itemize}
    \setlength{\itemsep}{0pt}
    \setlength{\parsep}{0pt}
    \setlength{\parskip}{0pt}
    \item \textbf{插入性能}:由于底层使用连续存储,在容器中间插入元素会导致后续元素移动,对于频繁插入删除的场景,传统的std::map可能更适合
    \item \textbf{迭代器失效}:任何修改容器的操作都可能导致迭代器失效,比传统map更需要注意
    \item \textbf{空间效率}:对于频繁增删的场景可能导致内存碎片
    \item \textbf{预分配优化}:对于已知大小的数据集,应考虑预先分配足够的空间
    \item \textbf{与排序算法配合}:可以考虑先收集所有键值对再一次性构建容器,而不是逐个插入
\end{itemize}

通过了解这些使用注意事项,开发者可以更合理地选择使用flat\_map或flat\_multimap的场景,并在具体实现中作出合适的优化决策。

\section{性能分析}

本节将从理论和实践两个方面对std::flat\_map与std::flat\_multimap的性能进行分析,并与传统的std::map和std::multimap进行对比。

\subsection{理论复杂度分析}

首先,从底层实现角度分析std::map/multimap与std::flat\_map/flat\_multimap的根本差异:

\begin{itemize}
    \setlength{\itemsep}{0pt}
    \setlength{\parsep}{0pt}
    \setlength{\parskip}{0pt}
    \item \textbf{std::map/multimap}: 通常基于红黑树(自平衡二叉搜索树)实现,每个键值对存储在独立的树节点中
    \item \textbf{std::flat\_map/flat\_multimap}: 使用两个平行数组(通常是std::vector)分别存储键和值,保持有序状态
\end{itemize}

这种根本实现差异导致了它们在各种操作上的时间复杂度差异:

\begin{center}\vspace{-20pt}
    \resizebox{\textwidth}{!}{
    \begin{tabular}{|l|c|c|c|}
    \hline
    \textbf{操作} & \textbf{std::map/multimap} & \textbf{std::flat\_map/flat\_multimap} & \textbf{底层原理} \\
    \hline
    查找 (find, contains) & $O(\log n)$ & $O(\log n)$ & 树的遍历 vs 二分查找 \\
    \hline
    插入 (insert) & $O(\log n)$ & $O(n)$ & 树的重平衡 vs 数组元素移动 \\
    \hline
    删除 (erase) & $O(\log n)$ & $O(n)$ & 树的重平衡 vs 数组元素移动 \\
    \hline
    遍历 (iteration) & $O(n)$ & $O(n)$ & 节点遍历 vs 连续内存访问 \\
    \hline
    \end{tabular}
    }
    \end{center}

详细分析各操作的底层实现差异:

\paragraph{查找操作} 虽然两种容器的查找操作理论复杂度都是$O(\log n)$,但实现方式不同:
\begin{itemize}
    \setlength{\itemsep}{0pt}
    \setlength{\parsep}{0pt}
    \setlength{\parskip}{0pt}
    \item std::map通过遍历树节点进行查找,每次比较后向左或向右子树移动
    \item std::flat\_map在有序数组上使用二分查找算法
    \item 虽然复杂度相同,但flat\_map的连续内存布局带来更好的缓存局部性,实际性能通常更优
\end{itemize}

\paragraph{插入操作} 两种容器在插入操作上的复杂度差异显著:
\begin{itemize}
    \setlength{\itemsep}{0pt}
    \setlength{\parsep}{0pt}
    \setlength{\parskip}{0pt}
    \item std::map需要找到正确位置($O(\log n)$)并可能执行树的重平衡($O(\log n)$),总体复杂度保持为$O(\log n)$
    \item std::flat\_map需要找到正确位置($O(\log n)$),然后可能需要移动后续元素以维持有序性($O(n)$),总体复杂度为$O(n)$
\end{itemize}

\paragraph{删除操作} 删除操作也存在类似的复杂度差异:
\begin{itemize}
    \setlength{\itemsep}{0pt}
    \setlength{\parsep}{0pt}
    \setlength{\parskip}{0pt}
    \item std::map删除节点后可能需要重平衡树结构,但总体保持$O(\log n)$复杂度
    \item std::flat\_map删除元素后需要移动后续元素填补空缺,这是一个$O(n)$操作
\end{itemize}

\paragraph{遍历操作} 虽然理论复杂度都是$O(n)$,但性能特性差异明显:
\begin{itemize}
    \setlength{\itemsep}{0pt}
    \setlength{\parsep}{0pt}
    \setlength{\parskip}{0pt}
    \item std::map的遍历涉及在内存中分散的节点间跳转,缓存命中率较低
    \item std::flat\_map遍历连续内存区域,缓存友好性显著更高,实际性能通常优于map
\end{itemize}

从理论复杂度分析可以看出:
\begin{itemize}
    \setlength{\itemsep}{0pt}
    \setlength{\parsep}{0pt}
    \setlength{\parskip}{0pt}
    \item 两种容器的查找操作都是对数复杂度,但flat\_map由于数据连续存储,缓存命中率更高
    \item flat\_map的插入和删除操作在最坏情况下需要线性时间,因为可能需要移动大量元素
    \item 遍历操作两者都是线性复杂度,但flat\_map由于数据连续存储,通常会更快
\end{itemize}

\subsection{实验设计}

为了验证理论分析并获取更直观的性能对比,我们设计了以下实验:

\begin{itemize}
    \setlength{\itemsep}{0pt}
    \setlength{\parsep}{0pt}
    \setlength{\parskip}{0pt}
    \item \textbf{实验1:查找性能测试} - 测量在不同大小的容器中执行随机查找操作的平均时间
    \item \textbf{实验2:插入性能测试} - 测量向已有容器中插入新元素的性能
    \item \textbf{实验3:批量构造测试} - 测量一次性构造大容量容器的性能
    \item \textbf{实验4:遍历性能测试} - 测量遍历容器中所有元素的时间
\end{itemize}

测试代码示例如下:

\begin{lstlisting}[style=cpp]
#include <chrono>
#include <iostream>
#include <map>
#include <flat_map>
#include <random>
#include <vector>
#include <iomanip>
#include <string>
#include <memory>

// 计时帮助函数
template<typename Func>
auto measure_time(Func&& func) {
    auto start = std::chrono::high_resolution_clock::now();
    func();
    auto end = std::chrono::high_resolution_clock::now();
    return std::chrono::duration<double, std::milli>(end - start).count();
}

template<typename Map, typename MultiMap>
void benchmark_containers(size_t size, size_t operation_count) {
    std::cout << "\n============================================\n";
    std::cout << "容器大小: " << size << ", 操作次数: " << operation_count << std::endl;
    std::cout << "============================================\n";
    
    // 准备随机数据
    std::random_device rd;
    std::mt19937 gen(rd());
    std::uniform_int_distribution<> dist(1, size * 10);
    
    // 创建测试用的键值对
    std::vector<std::pair<int, int>> data;
    data.reserve(size);
    for (size_t i = 0; i < size; ++i) {
        data.emplace_back(dist(gen), i);
    }
    
    // 准备查找和插入用的键
    std::vector<int> testKeys;
    testKeys.reserve(operation_count);
    for (size_t i = 0; i < operation_count; ++i) {
        testKeys.push_back(dist(gen));
    }
    
    // 测试结果结构
    struct TestResults {
        double lookup_time = 0;
        double insert_time = 0;
        double construction_time = 0;
        double iteration_time = 0;
        size_t memory_usage = 0;
    };
    
    TestResults map_results, flat_map_results;
    
    std::cout << "测试标准 map/multimap...\n";
    
    // 批量构造测试 - map
    map_results.construction_time = measure_time([&]() {
        Map regularMap(data.begin(), data.end());
    });
    
    // 创建测试实例
    Map regularMap;
    
    // 插入测试 - map
    map_results.insert_time = measure_time([&]() {
        for (size_t i = 0; i < operation_count; ++i) {
            regularMap.insert({testKeys[i], i});
        }
    });
    
    // 查找测试 - map
    size_t found = 0;
    map_results.lookup_time = measure_time([&]() {
        for (const auto& key : testKeys) {
            if (regularMap.find(key) != regularMap.end()) {
                ++found;
            }
        }
    });
    
    // 遍历测试 - map
    volatile int sum = 0;
    map_results.iteration_time = measure_time([&]() {
        for (const auto& [key, value] : regularMap) {
            sum += value;
        }
    });
    
    std::cout << "测试 flat_map/flat_multimap...\n";
    
    // 批量构造测试 - flat_map
    flat_map_results.construction_time = measure_time([&]() {
        MultiMap flatMap(data.begin(), data.end());
    });
    
    // 创建测试实例
    MultiMap flatMap;
    
    // 插入测试 - flat_map
    flat_map_results.insert_time = measure_time([&]() {
        for (size_t i = 0; i < operation_count; ++i) {
            flatMap.insert({testKeys[i], i});
        }
    });
    
    // 查找测试 - flat_map
    found = 0;
    flat_map_results.lookup_time = measure_time([&]() {
        for (const auto& key : testKeys) {
            if (flatMap.find(key) != flatMap.end()) {
                ++found;
            }
        }
    });
    
    // 遍历测试 - flat_map
    sum = 0;
    flat_map_results.iteration_time = measure_time([&]() {
        for (const auto& [key, value] : flatMap) {
            sum += value;
        }
    });

    // 输出结果
    std::cout << "----------- 性能对比 -----------\n";
    std::cout << std::fixed << std::setprecision(2);
    std::cout << std::left << std::setw(15) << "操作" 
            << std::setw(18) << "标准容器 (ms)" 
            << std::setw(18) << "flat容器 (ms)" 
            << std::setw(18) << "性能比" << std::endl;
            
    std::cout << std::left << std::setw(15) << "查找" 
            << std::setw(15) << map_results.lookup_time
            << std::setw(15) << flat_map_results.lookup_time
            << std::setw(15) << map_results.lookup_time / flat_map_results.lookup_time << std::endl;
            
    std::cout << std::left << std::setw(15) << "插入" 
            << std::setw(15) << map_results.insert_time
            << std::setw(15) << flat_map_results.insert_time
            << std::setw(15) << map_results.insert_time / flat_map_results.insert_time << std::endl;
            
    std::cout << std::left << std::setw(17) << "批量构造" 
            << std::setw(15) << map_results.construction_time
            << std::setw(15) << flat_map_results.construction_time
            << std::setw(15) << map_results.construction_time / flat_map_results.construction_time << std::endl;
            
    std::cout << std::left << std::setw(15) << "遍历" 
            << std::setw(15) << map_results.iteration_time
            << std::setw(15) << flat_map_results.iteration_time
            << std::setw(15) << map_results.iteration_time / flat_map_results.iteration_time << std::endl;
}

int main() {
    // 为了测试的完整性,测试不同大小的容器
    // 但较小的数据量以快速获得结果
    std::vector<size_t> sizes = {10000, 100000, 1000000};
    
    // 调整操作次数以平衡测试时间
    std::vector<size_t> operations = {10000, 100000, 1000000};
    
    for (size_t i = 0; i < sizes.size(); ++i) {
        size_t size = sizes[i];
        size_t ops = operations[i];
        
        std::cout << "\n===== Map vs Flat_map =====\n";
        benchmark_containers<std::map<int, int>, std::flat_map<int, int>>(size, ops);
        
        std::cout << "\n===== Multimap vs Flat_multimap =====\n";
        benchmark_containers<std::multimap<int, int>, std::flat_multimap<int, int>>(size, ops);
    }
    
    return 0;
}
\end{lstlisting}

运行结果如下:

\begin{lstlisting}[style=cpp]
===== Map vs Flat_map =====

============================================
容器大小: 10000, 操作次数: 10000
============================================
测试标准 map/multimap...
测试 flat_map/flat_multimap...
----------- 性能对比 -----------
操作         标准容器 (ms) flat容器 (ms)   性能比
查找         0.62           0.53           1.17
插入         0.63           1.63           0.39
批量构造     0.82           0.48           1.71
遍历         0.07           0.00           32.40

===== Multimap vs Flat_multimap =====

============================================
容器大小: 10000, 操作次数: 10000
============================================
测试标准 map/multimap...
测试 flat_map/flat_multimap...
----------- 性能对比 -----------
操作         标准容器 (ms) flat容器 (ms)   性能比
查找         0.66           0.54           1.23
插入         0.56           1.78           0.32
批量构造     0.69           0.41           1.68
遍历         0.08           0.00           32.27

===== Map vs Flat_map =====

============================================
容器大小: 100000, 操作次数: 100000
============================================
测试标准 map/multimap...
测试 flat_map/flat_multimap...
----------- 性能对比 -----------
操作         标准容器 (ms) flat容器 (ms)   性能比
查找         13.46          7.12           1.89
插入         10.67          146.62         0.07
批量构造     11.73          5.33           2.20
遍历         1.54           0.02           69.04

===== Multimap vs Flat_multimap =====

============================================
容器大小: 100000, 操作次数: 100000
============================================
测试标准 map/multimap...
测试 flat_map/flat_multimap...
----------- 性能对比 -----------
操作         标准容器 (ms) flat容器 (ms)   性能比
查找         14.47          7.17           2.02
插入         9.74           160.01         0.06
批量构造     10.88          5.05           2.15
遍历         1.65           0.02           70.24

===== Map vs Flat_map =====

============================================
容器大小: 1000000, 操作次数: 1000000
============================================
测试标准 map/multimap...
测试 flat_map/flat_multimap...
----------- 性能对比 -----------
操作         标准容器 (ms) flat容器 (ms)   性能比
查找         824.18         104.21         7.91
插入         471.17         17398.70       0.03
批量构造     481.34         60.37          7.97
遍历         137.53         0.20           684.66

===== Multimap vs Flat_multimap =====

============================================
容器大小: 1000000, 操作次数: 1000000
============================================
测试标准 map/multimap...
测试 flat_map/flat_multimap...
----------- 性能对比 -----------
操作         标准容器 (ms) flat容器 (ms)   性能比
查找         803.13         98.35          8.17
插入         451.21         17175.69       0.03
批量构造     423.51         54.42          7.78
遍历         130.15         0.23           555.74
\end{lstlisting}

考虑到map组和multimap组的实验结果类似,下面只展开分析map组的实验结果。

\subsection{实验结果}

\subsubsection{查找性能}

在查找操作测试中,我们观察到以下结果:

\begin{center}\vspace{-20pt}
    \resizebox{\textwidth}{!}{
    \begin{tabular}{|c|c|c|c|}
    \hline
    \textbf{容器大小} & \textbf{std::map 耗时(ms)} & \textbf{std::flat\_map 耗时(ms)} & \textbf{性能比} \\
    \hline
    10,000 & 0.62 & 0.53 & 1.17x \\
    \hline
    100,000 & 13.46 & 7.12 & 1.89x \\
    \hline
    1,000,000 & 824.18 & 104.21 & 7.91x \\
    \hline
    \end{tabular}
    }
\end{center}

在查找操作中,std::flat\_map表现出明显的性能优势,尤其是在数据量较大时。这主要得益于数据的连续存储带来的缓存友好性,减少了缓存未命中的情况。

\subsubsection{插入性能}

对于插入操作,我们发现以下趋势:

\begin{center}\vspace{-20pt}
    \resizebox{\textwidth}{!}{
    \begin{tabular}{|c|c|c|c|}
    \hline
    \textbf{容器大小} & \textbf{std::map 耗时(ms)} & \textbf{std::flat\_map 耗时(ms)} & \textbf{性能比} \\
    \hline
    10,000 & 0.63 & 1.63 & 0.39x \\
    \hline
    100,000 & 10.67 & 146.62 & 0.07x \\
    \hline
    1,000,000 & 471.17 & 17398.70 & 0.03x \\
    \hline
    \end{tabular}
    }
\end{center}

在插入操作中,std::flat\_map的性能明显劣于std::map,尤其是在数据量较大时。这主要是因为每次插入可能需要移动大量元素,导致性能下降。

\subsubsection{批量构造性能}

对于一次性构造大量数据的场景,结果如下:

\begin{center}\vspace{-20pt}
    \resizebox{\textwidth}{!}{
    \begin{tabular}{|c|c|c|c|}
    \hline
    \textbf{数据量} & \textbf{std::map 耗时(ms)} & \textbf{std::flat\_map 耗时(ms)} & \textbf{性能比} \\
    \hline
    10,000 & 0.82 & 0.48 & 1.71x \\
    \hline
    100,000 & 11.73 & 5.33 & 2.20x \\
    \hline
    1,000,000 & 481.34 & 60.37 & 7.97x \\
    \hline
    \end{tabular}
    }
\end{center}

在批量构造场景中,std::flat\_map表现出明显的性能优势,这主要是因为:
\begin{itemize}
    \setlength{\itemsep}{0pt}
    \setlength{\parsep}{0pt}
    \setlength{\parskip}{0pt}
    \item 一次性分配内存减少了频繁分配的开销
    \item 减少了节点创建和链接的开销
\end{itemize}

\subsubsection{遍历性能}

遍历测试显示:

\begin{center}\vspace{-20pt}
    \resizebox{\textwidth}{!}{
    \begin{tabular}{|c|c|c|c|}
    \hline
    \textbf{容器大小} & \textbf{std::map 耗时(ms)} & \textbf{std::flat\_map 耗时(ms)} & \textbf{性能比} \\
    \hline
    10,000 & 0.07 & 0.00 & 32.40x \\
    \hline
    100,000 & 1.54 & 0.02 & 69.04x \\
    \hline
    1,000,000 & 137.53 & 0.20 & 684.66x \\
    \hline
    \end{tabular}
    }
\end{center}

在遍历操作中,std::flat\_map的性能优势非常明显,可达std::map的数十到上百倍。这主要得益于数据的连续存储带来的缓存友好性,减少了CPU等待时间。

\subsection{分析与讨论}

基于以上实验结果,我们可以得出以下结论:

\begin{enumerate}
    \setlength{\itemsep}{0pt}
    \setlength{\parsep}{0pt}
    \setlength{\parskip}{0pt}
    \item \textbf{查找操作}:std::flat\_map在查找操作上表现优异,这主要得益于数据连续存储带来的缓存友好性,减少了缓存未命中的情况。
    
    \item \textbf{插入操作}:在大数据量场景下,std::flat\_map的插入性能显著劣于std::map。这与理论复杂度分析一致,因为每次插入可能需要移动大量元素。
    
    \item \textbf{批量构造}:当需要一次性构建容器时,std::flat\_map表现更好,因为它减少了频繁内存分配和节点创建的开销。
    
    \item \textbf{遍历性能}:std::flat\_map的遍历性能可达std::map的数十到上百倍,这是最显著的性能优势之一,非常适合频繁遍历的场景。
\end{enumerate}

综合性能测试表明,std::flat\_map在查找、遍历和批量构造方面具有明显优势,但在频繁单次插入删除的场景下,std::map仍然是更好的选择。这符合我们对其底层实现机制的理论分析。

\section{优缺点与使用场景}

基于前述的理论分析和实验结果,本节将总结std::flat\_map和std::flat\_multimap的优缺点,并探讨其最佳使用场景,帮助开发者在实际应用中做出明智的选择。

\subsection{优点}

std::flat\_map和std::flat\_multimap相比传统的map容器具有以下优势:

\begin{itemize}
    \setlength{\itemsep}{0pt}
    \setlength{\parsep}{0pt}
    \setlength{\parskip}{0pt}
    \item \textbf{卓越的查找性能}:实验数据表明,flat\_map的查找操作比传统map快7倍,尤其在大型数据集上优势更为明显。
    
    \item \textbf{显著的遍历性能优势}:得益于连续内存布局,flat\_map的遍历速度可达传统map的数十到上百倍,这是其最突出的性能优势之一。
    
    \item \textbf{高效的批量构造}:在一次性构建大容器的场景中,flat\_map比传统map快约7倍。
    
    \item \textbf{优秀的缓存局部性}:数据存储在连续内存区域,显著提高了缓存命中率,减少了CPU等待时间。
    
    \item \textbf{熟悉的API接口}:与传统map保持高度相似的接口,降低了学习成本和代码迁移难度。
    
    \item \textbf{底层容器可访问性}:提供了直接访问和操作底层容器的方法,增加了灵活性。
\end{itemize}

\subsection{缺点}

然而,flat\_map也存在一些明显的不足:

\begin{itemize}
    \setlength{\itemsep}{0pt}
    \setlength{\parsep}{0pt}
    \setlength{\parskip}{0pt}
    \item \textbf{插入性能劣势}:在大型容器中,单次插入操作的性能可能比传统map慢几百倍,实验中在100万元素的容器中,差距达到30倍。
    
    \item \textbf{删除操作效率低}:删除元素后需要移动后续所有元素,在大数据集上开销显著。
    
    \item \textbf{迭代器易失效}:任何修改容器的操作都可能导致迭代器、引用和指针失效,程度超过传统map。
    
    \item \textbf{潜在的内存碎片}:频繁的插入删除操作可能导致vector频繁重新分配,造成内存碎片。
    
    \item \textbf{实现相对新}:作为C++23的新特性,部分编译器和库可能尚未完全支持或优化。
\end{itemize}

\subsection{适用场景}

基于以上分析,std::flat\_map和std::flat\_multimap特别适合以下应用场景:

\begin{itemize}
    \setlength{\itemsep}{0pt}
    \setlength{\parsep}{0pt}
    \setlength{\parskip}{0pt}
    \item \textbf{读操作远多于写操作的场景}:如配置数据、常量表、字典等查询频繁但修改罕见的数据结构。
    
    \item \textbf{需要频繁遍历的应用}:如游戏引擎中的实体系统、批处理操作、图形渲染管线等需要高效遍历所有元素的场景。
    
    \item \textbf{内存受限的环境}:嵌入式系统、移动设备等对内存占用敏感的平台。
    
    \item \textbf{一次构建多次使用的静态数据}:如预计算的查找表、静态资源索引等。
    
    \item \textbf{批量操作场景}:当可以将多次修改合并为一次批量操作时,可以有效规避单次插入的性能劣势。
    
    \item \textbf{对缓存友好性要求高的性能关键代码}:如高性能计算、实时系统等对延迟敏感的场景。
\end{itemize}

\subsection{不适用场景}

相反,在以下场景中应当避免使用flat\_map:

\begin{itemize}
    \setlength{\itemsep}{0pt}
    \setlength{\parsep}{0pt}
    \setlength{\parskip}{0pt}
    \item \textbf{频繁单次插入或删除的动态集合}:如不断添加和移除元素的事件队列、日志系统等。
    
    \item \textbf{大型数据集且需要频繁修改}:元素数量庞大且经常变动的数据结构。
    
    \item \textbf{依赖迭代器稳定性的算法}:需要在遍历过程中修改容器的场景,迭代器易失效会带来问题。
    
    \item \textbf{并发修改频繁的共享数据}:在多线程环境中,线性时间的修改操作可能导致更长的锁定时间。
\end{itemize}

\subsection{优化策略}

在决定使用flat\_map后,可以考虑以下优化策略以获得最佳性能:

\begin{itemize}
    \setlength{\itemsep}{0pt}
    \setlength{\parsep}{0pt}
    \setlength{\parskip}{0pt}
    \item \textbf{预分配容量}:对于可预知大小的数据集,应通过底层容器预分配足够的空间以减少重分配开销。
    
    \item \textbf{批量构建}:尽可能收集所有元素后一次性构建容器,而非逐个添加。
    
    \item \textbf{利用特有操作}:使用extract()和replace()等特有操作对底层容器进行批量修改,可以避免渐进式修改的开销。
    
    \item \textbf{考虑自定义容器}:在特殊场景下,可以考虑为KeyContainer和MappedContainer使用std::deque或其他适合的容器类型,平衡不同操作的性能。
\end{itemize}

总之,std::flat\_map和std::flat\_multimap为C++开发者提供了在特定场景下性能优越的关联容器选择。通过深入了解其特性和适用场景,开发者可以在实际应用中充分利用其优势,同时规避其局限性,从而开发出更高效的软件系统。

\section{总结}

本文对C++23标准新引入的std::flat\_map和std::flat\_multimap两种容器进行了系统性探究,从基本用法、使用案例到性能分析,全面审视了这两种新型关联容器的特性与价值。通过对比研究,我们得出以下结论:

std::flat\_map和std::flat\_multimap采用平行数组作为底层存储结构,与传统的基于树的map和multimap相比,在访问模式和性能特征上有显著差异。它们表现出连续内存布局的典型优势,在查找和遍历方面表现卓越,而在动态插入删除操作上则存在明显劣势。

性能测试表明,这两种新容器在查找和批量构造操作上比传统map快7倍,并在遍历操作上快数十到上百倍,。这种性能特点使其特别适合以下场景:读多写少的应用以及需要频繁遍历的系统。

然而,在频繁单次插入删除、大型数据集动态修改等场景中,传统map的对数复杂度优势仍然不可替代。这表明C++标准库朝着“为不同场景提供专门工具”的方向发展,而非追求单一通用解决方案。

对于开发者而言,理解这些新容器的内部机制至关重要,这有助于:
\begin{itemize}
    \setlength{\itemsep}{0pt}
    \setlength{\parsep}{0pt}
    \setlength{\parskip}{0pt}
    \item 在性能关键场景中做出正确的容器选择
    \item 针对所选容器的特性优化代码
    \item 更好地理解性能与内存使用之间的权衡
\end{itemize}

随着C++23标准的逐步普及,std::flat\_map和std::flat\_multimap有望在特定应用领域获得广泛应用,特别是在游戏开发、嵌入式系统、高性能计算等对性能和内存效率有严格要求的场景。未来编译器和库实现的进一步优化,也有可能缓解这些容器在修改操作上的性能劣势。

总之,std::flat\_map和std::flat\_multimap是C++标准库的重要补充,它们不是用来取代传统map的工具,而是为开发者提供了更多元化的选择,使我们能够根据特定应用需求选择最合适的容器,从而开发出更高效的软件系统。正如本文分析所示,了解这些工具的优缺点和适用场景,对于充分发挥C++语言的性能潜力至关重要。

%\section*{参考文献}

\begin{thebibliography}{99}

\bibitem{cpp23std}
ISO/IEC. \emph{Programming Languages -- C++}, ISO/IEC 14882:2023(E), 2023.

\bibitem{boost}
Boost C++ Libraries. \emph{Boost.Container: flat\_map and flat\_set}, 2023.
\\\url{https://www.boost.org/doc/libs/release/doc/html/container/non_standard_containers.html}

\bibitem{cppreference}
cppreference.com. \emph{std::flat\_map, std::flat\_multimap}, 2023.
\\\url{https://en.cppreference.com/w/cpp/container/flat_map}

\end{thebibliography}

\end{document}